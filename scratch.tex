

%Dans le cas d'un modèle non-paramétrique, l'intégrale de l'équation \ref{eq:panco:y_abel} est la somme des intégrales des profils dans chaque intervalle de rayon.
%L'équation \ref{eq:panco:nonparam} peut se décomposer en trois cas distincts :
%
%\begin{itemize}[leftmargin=*]
%\setlength\itemsep{5pt}
%
%\item Cas 1 (C1) : Si $r < R_0$, où $R_0$ est le rayon du premier point de pression,
%\begin{equation}
%    P^{\rm C1}(r) = \bigg\{\begin{array}{ll}
%        P_0 (r/R_0)^{- \alpha_0} & \text{pour } r \leqslant R_0; \\
%        0                        & \text{pour } r > R_0.
%        \end{array}
%\end{equation}
%
%\item Cas 2 (C2) : Si $R_i < r < R_{i+1}$, c'est-à-dire aux rayons contenus entre deux points de mesure du profil de pression,
%\begin{equation}
%    P^{\rm C2}(r) = \Bigg\{\begin{array}{ll}
%        0                        & \text{pour } r < R_i; \\
%        P_i (r/R_i)^{- \alpha_i} & \text{pour } R_i \leqslant r \leqslant R_{i+1}; \\
%        0                        & \text{pour } r > R_{i+1}.
%        \end{array}
%\end{equation}
%
%\item Cas 3 (C3) : Si $r > R_n$, où $R_n$ est le rayon du dernier point de pression,
%\begin{equation}
%    P^{\rm C3}(r) = \bigg\{\begin{array}{ll}
%        0                        & \text{pour } r < R_n; \\
%        P_n (r/R_n)^{- \alpha_0} & \text{pour } r \geqslant R_n.
%        \end{array}
%\end{equation}
%\end{itemize}
%
%L'intégrale le long de la ligne de visée de chacune de ces fonctions s'écrit donc
%\begin{equation}
%    \mathds{I} = \int_{-z_0}^{+z_0} P(r) \d z
%\end{equation}
%et peux s'exprimer en faisant apparaître la fonction bêta incomplète régularisée $I(x; a, b)$:
%
%\begin{equation}
%    \mathds{I}^{\rm C1}(r) = \bigg\{\begin{array}{ll}
%    \kappa - \kappa I (x^2; \; q_i, \; 1/2) & \text{pour } x \equiv r/R_0 \leqslant 1; \\
%    0                                    & \text{pour } x > 1.
%        \end{array}
%\end{equation}
%\begin{equation}
%    \mathds{I}^{\rm C2}(r) = \bigg\{\begin{array}{ll}
%    \kappa I (x^2; \; q_i, \; 1/2) & \text{pour } x \equiv r/R_0 \leqslant 1; \\
%    0                                    & \text{pour } x > 1.
%        \end{array}
%\end{equation}



% ------------------------------------------------------------------------------------ %
%\subsection{Observations en X et masse hydrostatique}
%
%L'une des grandeurs nécessaires pour la cosmologie avec des amas est la masse de chacun des amas de l'échantillon considéré.
%Il existe plusieurs façons de mesurer la masse d'un amas (comme décrit dans le chapitre \addref), parmi lesquelles la mesure de ``masse hydrostatique'' : dans l'hypothèse de l'équilibre hydrostatique, la masse contenue dans une sphère de rayon $r$ peut etre reliée aux propriétés thermodynamiques par
%
%\begin{equation}
%    \label{eq:panco:mhse}
%    M_\mathrm{HSE}(r) = -\frac{1}{\mu m_{\rm p} G} \frac{r^2}{n_{\rm e}(r)} \frac{\mathrm{d}P_{\rm e}}{\mathrm{d} r},
%\end{equation}
%
%où  $m_{\rm p}$ est la masse du proton, $G$ la constante universelle de gravitation, $\mu$ représente le poids moléculaire moyen du milieu intra-amas, et $n_{\rm e}$ et $P_{\rm e}$ sa densité et pression d'électrons.
%La densité d'électrons mesurée à l'aide d'observations de l'emission X du mileu intra-amas peut alors etre combinée au profil de pression mesuré en SZ pour obtenir la distribution radiale de masse d'un amas.

% =====================================================================================
% =====================================================================================

%Cette technique est basée sur une division des dimensions de l'espace des paramètres en plusieurs groupes de paramètres; par exemple, un vecteur de paramètres de dimension 100 peut être divisé en vingt sous-vecteurs de dimension 5.
%L'échantillonnage est ensuite réalisé bloc par bloc, avec une probabilité d'acceptance conditionnelle sur le groupe précédent.
%Pour l'échantillonnage d'un espace à $N$ dimensions, l'algorithme peut être résumé par la séquence suivante:
%\begin{enumerate}[leftmargin=*]
%    \item Une division des paramètres en $d$ groupes de dimensions $n_d$ est réalisée:
%    \begin{equation}
%        \vartheta = (\vartheta_1, \dots \vartheta_N)
%        \rightarrow \theta = (\theta_1, \dots \theta_d)
%        = (\{\vartheta_1, \dots \vartheta_{n_1}\}, \{\vartheta_{n_1 + 1} \dots \vartheta_{n_1 + n_2}\}, \dots \{\vartheta_{N - n_d}, \dots \vartheta_N\}).
%    \end{equation}
%    \item La marche aléatoire commence en un point de l'espace des paramètres de coordonnées définies par un vecteur $\theta$.
%    \item Pour le premier groupe $\theta_1$, une position proche dans l'espace des paramètres, $\theta_1'$, est tirée au hasard d'après une fonction de proposition prédéfinie.
%    \item La même procédure est répétée pour chacun des groupes de paramètres.
%        Pour un groupe $i$, la fonction de proposition dépend des positions de tous les autres groupes:
%
%
%
%
%
%
%    \item Une position proche dans l'espace des paramètres, $\vartheta'$, est tirée au hasard d'après une fonction de proposition prédéfinie.
%    \item Si $P(\vartheta') \geqslant P(\vartheta)$, alors $\vartheta'$ est automatiquement acceptée comme nouvelle position de l'échantillonneur ; sinon, elle est aléatoirement acceptée ou non avec une probabilité d'autant plus faible que $P(\vartheta') < P(\vartheta)$, nommée probabilité d'acceptance.
%    \item La procédure est répétée autant de fois que nécessaire pour obtenir un nombre d'échantillons jugé suffisant.
%\end{enumerate}

% =====================================================================================
% =====================================================================================

%\begin{table}[t]
%    \setlength{\tabcolsep}{5pt}
%    \small
%    \centering
%    \begin{tabular}{c c c c c c}
%        \toprule
%        \multirow{2}{*}{Paramètre $\vartheta$} & \multicolumn{2}{c}{Paramètres de l'analyse} & \multicolumn{3}{c}{Résultats} \\
%        \cmidrule(lr){2-3} \cmidrule(lr){4-6}
%        & Valeur d'entrée & Distribution \prior
%        & $\xi_\vartheta \;[\%]$ & $\zeta_\vartheta \; [\sigma]$ & $\eta_\vartheta \; [\%]$ \\
%        \midrule
%        % -------------------------------------------------------------------------- %
%        $\alpha_{Y|Z}$ & $1   $ & $\mathcal{U}(-10^4, 10^4)$
%                     & $0.27$  & $0.00$  & $2.11$ \\
%        $\beta_{Y|Z}$  & $1   $ & Student $t_1$
%                     & $-0.99$  & $-0.01$  & $2.39$ \\
%        $\sigma_{Y|Z}$ & $0.5 $ & $\mathcal{U}(0, 10^4)$
%                     & $-4.94$  & $-0.13$  & $43.70$ \\
%        % -------------------------------------------------------------------------- %
%        $\gamma$       & $-2/3$ & $\delta(-2/3)$ & -- & -- &  -- \\
%        $\delta$       & $0   $ & $\delta(0)$ & -- & -- & -- \\
%        $\alpha_{X|Z}$ & $0   $ & $\delta(0)$ & -- & -- & -- \\
%        $\beta_{X|Z}$  & $1   $ & $\delta(1)$ & -- & -- & -- \\
%        $\sigma_{X|Z}$ & $0.1 $ & $\mathcal{U}(0, 10^4)$ & -- & -- & -- \\
%        \midrule
%        \bottomrule
%    \end{tabular}
%    \caption{%
%        Liste des valeurs d'entrée des paramètres, de leurs distributions \prior\ et des résultats pour l'analyse présentée en section \ref{sec:scaling:valid2}.
%        La distribution \prior\ des paramètres non-répertoriés dans cette table est celle par défaut (table \ref{tab:scaling:params}).
%    }
%    \label{tab:scaling:valid2}
%\end{table}

