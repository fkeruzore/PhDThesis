Le travail présenté dans ce manuscrit est le fruit de trois années de doctorat, préparé au Laboratoire de Physique Subatomique et Cosmologie (LPSC) de Grenoble, au sein de l'équipe Cosmologie Multilongueur d'onde.
Je tiens tout d'abord à remercier Arnaud Lucotte, directeur du LPSC, pour m'avoir accueilli au sein du laboratoire, et permis d'y travailler dans les meilleures conditions.

Un grand merci à tous les membres de mon jury de thèse.
Merci à Delphine Hardin d'en avoir assuré la présidence.
Merci à Sophie Maurogordato et Jean-Baptiste Melin d'avoir accepté d'être rapporteurs, pour leur lecture attentive du manuscrit, et leurs retours très constructifs.
Merci également aux examinateurs, Dominique Boutigny et Pierre Salati, pour l'intérêt que vous avez porté à mon travail.
Merci à tou$\cdot$te$\cdot$s d'avoir accepté de participer à la soutenance, pour vos retours, et les discussions qui ont suivi.

Un immense merci à Frédéric Mayet, pour avoir été le meilleur directeur de thèse du monde\textsuperscript{\textsc{tm}}.
Pendant trois ans, tu as toujours été présent pour me guider et me conseiller, à chaque fois que j'en ai eu besoin, tout en m'accordant ta confiance.
Merci pour tout le temps que tu m'as consacré, pour tout le soutien que tu m'as apporté, pour tout ce que tu m'as appris -- de la physique à son enseignement, en passant par l'organisation de la recherche et l'histoire du Dauphiné --, pour ta patience, ton amitié, et plus généralement pour le climat de travail que tu m'as offert.
Préparer une thèse sous ta direction est une chance immense, et le meilleur qu'un étudiant puisse souhaiter.

Merci aux membres de l'équipe Cosmologie Multi-$\lambda$, a.k.a. \guillemotleft Cosmo-Mleuh \guillemotright, pour avoir offert pendant trois ans un cadre de travail plein d'entraide et de soutien constant, dans une bonne humeur perpétuelle.
Mille mercis à Juan \textit{jefe} Mac\'ias-P\'erez, pour toute l'expertise que tu m'as communiquée avec énormément de patience, de l'instrumentation à la cosmologie; pour ta gentillesse, et pour avoir été toujours disponible.
\textit{Aunque no quieras que te llamemos jefe, ¡ sigues siendo el mejor que hay!}
Un grand merci à Florian \guillemotleft 1 \guillemotright\ Ruppin, pour toute l'aide que tu m'as apportée au cours de ma thèse, des conseils du début il y a trois ans à l'aide précieuse dans la recherche de post-docs, en passant par toutes les discussions sur l'exploitation des données du LPSZ et autres.
Ton soutien a été très précieux au long de ma thèse, et j'espère que nous aurons l'occasion de retravailler ensemble très vite.
Merci à Laurence Perotto pour toutes nos discussions sur l'analyse des données SZ de NIKA2, et pour m'avoir expliqué tant de fois la calibration sans jamais perdre patience (bon, sauf une fois -- \guillemotleft Vous avez pas un bureau? \guillemotright).
Merci à Nicolas Ponthieu pour tout le temps passé sur la réduction des données brutes de NIKA2, pour ton aide avec \texttt{IDL}, et pour les débats enflammés sur les éditeurs de texte.
Je suis sûr qu'avec un an de thèse de plus, je serais arrivé à te faire entendre raison; nous ne saurons jamais.
Merci à Andrea Catalano et François-Xavier Désert pour les discussions sur l'instrumentation, les KID, et les instruments CMB post-NIKA2.
Merci à Alex pour avoir été un super co-bureau tout au long de nos thèses, et pour tous les rires partagés.
Merci à Miren et Emmanuel pour toutes nos conversations, pour vos retours précieux sur le travail de ma thèse, pour ne jamais avoir craqué face aux bugs de PANCO2, et pour une affiche de thèse historique!
Merci à Corentin, le dernier arrivé, à qui je souhaite une excellente thèse dans cette excellente équipe.

Je tiens également à remercier l'ensemble de la collaboration NIKA2, au sein de laquelle j'ai énormément appris au cours de ma thèse.
Merci aux membres du LPSZ de NIKA2, et en particulier à Monique Arnaud, Etienne Pointecouteau, Gabriel Pratt, Jean-Baptiste Melin et Iacopo Bartalucci.
Votre expertise sur les propriétés des amas de galaxies et sur l'analyse des observations X et SZ a été très précieuse, et j'ai beaucoup appris au cours de nos réunions.
Merci aux collègues de Rome et Madrid, en particulier Marco De Petris et Gustavo Yepes.
Les discussions très intéressantes et différents travaux sur les projets MUSIC et Three Hundred m'ont beaucoup appris sur les simulations hydrodynamiques, et sur les amas de galaxies en général.
Merci à l'IRAM de m'avoir permis d'observer au télescope de 30 mètres, ce qui m'a permis de comprendre plus profondément de fonctionnement des observations dans un cadre idéal.
Merci à toutes celles et ceux avec qui j'ai eu la chance d'observer au télescope, que je n'énumèrerai pas, par peur d'en oublier.
Merci en particulier à Bilal, aux côtés de qui j'ai énormément appris sur NIKA2 et le télescope, mais aussi sur la distribution des bars à tapas dans Grenade, toujours dans une atmosphère très détendue sur fond de \guillemotleft MarbleLympics \guillemotright.
Merci aussi à Alexandre pour les discussions très animées sur l'analyse de données et sur Python, au détour desquelles j'ai aussi beaucoup appris.
Merci enfin au jambon du télescope, pour le soutien et le réconfort qu'il m'a procuré au cours des longues nuits d'observation.

Merci également à tous les collègues du troisième étage du LPSC.
En particulier, je remercie les membres du groupe DARK pour autant de discussions scientifiques intéressantes que de rires partagés pendant les pauses café et déjeuner.
Merci à David pour ton investissement dans le suivi de thèse, et pour nos conversations très intéressantes sur l'analyse de données.
Merci aussi à Céline pour les discussions sur LSST et les amas dans les relevés optiques.
Merci à tous les deux pour l'ambiance générale et la bonne humeur que vous avez insufflées dans les pauses et repas de ces trois ans.
Merci à Laurent pour nos conversations sur l'enseignement et, à ma grande surprise, sur le Lot-et-Garonne.
Merci à vous tous, ainsi qu'à Constantin, Calum, Johan, et aux membres des groupes ATLAS, ALICE, MIMAC, pour tous ces moments partagés.
Merci à Cécile, pour sa gentillesse infinie qui nous manque tellement, et à qui j'aurais tellement voulu pouvoir dire \guillemotleft Ça y est, je suis un grand, je suis docteur! \guillemotright.

À tous les doctorants et post-doc du LPSC, pour l'ambiance du BIDUL.
Merci à Aina pour tous les rires partagés dans le bureau 333.
À Nathan, pour m'avoir suivi de ville en ville, offrant toujours un visage familier avec qui parler de foot  -- je t'attends à Chicago!
À Killian, le roi des boucles (ni doctorant ni post-doc, mais membre du BIDUL \textit{honoris causa}), pour les discussions intéressantes, en particulier en backstage lors des conférences en visio.
À Raphaël, le capital \blackout{qwertyuiopasdfhjkl}, pour ta sympathie du quotidien.
Merci à vous, et à tous les autres, pour les repas, rires, et verres partagés.

Merci aux amis de Grenoble et d'ailleurs pour leur soutien.
À Kim, Aline et Aël, je vous souhaite le meilleur dans votre nouvelle aventure; \textit{lycka till!}
À Jhon, pour ta présence constante, même quand je pars à 7000 km; \textit{you are the friend I need, not the one I deserve}.
Aux Marseillais, Montpelliérains, Bordelais, et Marmandais, qui se reconnaîtront.
À Helga, pour le soutien moral apporté dans le bureau 333.
Au carnet, pour la même raison.
À ma famille, pour m'avoir toujours encouragé et soutenu, même dans les moments les plus difficiles, merci pour tout.

À tous ceux que j'ai certainement oublié, toutes mes excuses!
J'emprunte une idée à \cite{maurin_propagation_2001} pour réparer ce tort, et que chacun$\cdot$e puisse être inclus$\cdot$e: \\
Enfin, je remercie \fbox{\phantom{Florian Kéruzoré}}, pour \fbox{\phantom{avoir existé aussi longtemps}}, et sans qui rien de tout cela n'aurait été possible.
